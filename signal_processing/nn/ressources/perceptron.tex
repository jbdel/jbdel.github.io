\documentclass{article}

\usepackage{tikz}
\begin{document}
\pagestyle{empty}

\def\layersep{2.5cm}

\begin{tikzpicture}[shorten >=1pt,->,draw=black!50, node distance=\layersep]
    \tikzstyle{every pin edge}=[<-,shorten <=1pt]
    \tikzstyle{neuron}=[circle,fill=black!25,minimum size=17pt,inner sep=0pt]
    \tikzstyle{input neuron}=[neuron, fill=green!50];
    \tikzstyle{output neuron}=[neuron, fill=red!50];
    \tikzstyle{hidden neuron}=[neuron, fill=blue!50];
    \tikzstyle{annot} = [text width=4em, text centered]

    % Draw the input layer nodes
    \foreach \name / \y in {1,...,4}
    % This is the same as writing \foreach \name / \y in {1/1,2/2,3/3,4/4}
        \node[input neuron, pin=left:$x$\y] (I-\name) at (0,-\y) {};

            
    % Draw the output layer node
    \node[output neuron,pin={[pin edge={->}]right:Output}, right of=I-2] (O) {};


            
    % Connect every node in the hidden layer with the output layer
    \foreach \source in {1,...,4}
        \path (I-\source) edge (O);

    % Annotate the layers
    \node[annot,above of=I-1, node distance=1cm] (hl) {Input};
    \node[annot,right of=hl] {Perceptron};
    
\end{tikzpicture}
% End of code
\end{document}