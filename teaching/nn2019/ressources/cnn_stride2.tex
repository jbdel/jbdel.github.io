
\documentclass{article}

\usepackage{tikz}
\begin{document}
\pagestyle{empty}

\def\layersep{2.5cm}

\begin{tikzpicture}[shorten >=1pt,->,draw=black!50, node distance=\layersep]
    \tikzstyle{every pin edge}=[<-,shorten <=1pt]
    \tikzstyle{neuron}=[circle,fill=black!25,minimum size=17pt,inner sep=0pt]
    \tikzstyle{input neuron}=[neuron, fill=green!50];
    \tikzstyle{output neuron}=[neuron, fill=red!50];
    \tikzstyle{hidden neuron}=[neuron, fill=blue!50];
    \tikzstyle{annot} = [text width=4em, text centered]

    % Draw the input layer nodes
    \foreach \name / \y in {1/0,2/1,3/2,4/-1,5/1,6/-3,7/0}
    % This is the same as writing \foreach \name / \y in {1/1,2/2,3/3,4/4}
        \node[input neuron, pin=left:\y] (I-\name) at (0,-\name) {};

    % Draw the hidden layer nodes
    \foreach \name / \y in {1/-2,2/1,3/1}
        \path[yshift=-2.0cm]
            node[hidden neuron, pin={[pin edge={->}]right:\y}] (H-\name) at (\layersep,-\name cm) {};


    % Draw the output layer node
    \node[output neuron,pin={[pin edge={-}]right:Filtre}, right of=H-1] (O) {};



    \path (I-1) edge (H-1);
    \path (I-2) edge (H-1);
    \path (I-3) edge (H-1);

    \path (I-3) edge (H-2);
    \path (I-4) edge (H-2);
    \path (I-5) edge (H-2);

    \path (I-5) edge (H-3);
    \path (I-6) edge (H-3);
    \path (I-7) edge (H-3);
        


    % Annotate the layers
    %\node[annot,above of=H-1, node distance=1cm] (hl) {Hidden layer 1};
    %\node[annot,left of=hl] {Input layer};
    %\node[annot,right of=hl] {Output layer};
    \node[annot,above of=I-1, node distance=1cm](hl) {Input};
    \node[annot,right of=hl] {Result};
    \node[annot,below of=O, node distance=1cm] {[1, 0, -1] Stride 1};


\end{tikzpicture}
% End of code
\end{document}
